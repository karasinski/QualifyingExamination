%!TEX root = ../main.tex
\documentclass[float=false, crop=false]{standalone}
\usepackage[subpreambles=true]{standalone}
\begin{document}

\section{Proposed Research}

% \subsection{Summary}

Concurrent bandwidth feedback has been used in a large variety of motor control tasks, and has generally been found to improve performance.
Until recently, however, only simple tasks such as physical movements or basic pursuit tasks have been investigated.
More recent works, including the lane-keeping task by de Groot et al., and our previous work with the SAFER task, have indicated that concurrent bandwidth feedback can also be quite effective for complex tasks.
The decrease in required learning time, improved performance, and decreased workload seen in the SAFER task show that concurrent bandwidth feedback may prove to be most useful very early in training when subjects are first exposed to complex, highly dynamic tasks.
% While visual concurrent bandwidth feedback has been used in a variety of tasks, no researchers have investigated its effects on a three-axis tracking task.
If concurrent bandwidth can improve performance without an increase in workload, then it may prove a useful technique for training other robotics tasks.

% Despite extensive previous research, to the authors' knowledge there exists no study in the literature addressing human performance or workload changes in manual tracking tasks between traditional computer monitors and mobile, augmented reality headsets.
% If operator performance while using augmented reality displays is improved--or at the very least, not degraded--then these devices could prove especially valuable in scenarios where it is impractical or otherwise difficult to provide a traditional computer interface.
% There are a variety of robotics tasks, such as pick-and-place tasks, for which performance may be improved by allowing an operator the mobility to move and view the scene from whatever position is convenient at a given time.
% Traditional robotics stations require the operator to remain in a single position, and typically only allow for several camera angles.
% Mobile augmented reality displays allow the operator to take advantage of their ability to move through the environment, without needing to manage external cameras.

There has been considerable improvement in the field of pilot modeling since McRuer's crossover model, especially with models that incorporate human physiology.
The Structural Model, in particular, has been very effective in predicting pilot performance, handling qualities, pilot-induced oscillation rating levels, and workload for a variety of system dynamics.
None of these pilot models, however, are able to include the effects of concurrent bandwidth feedback.
The performance improving effects of this feedback, seen throughout the literature, make this a compelling feature to be incorporated into a pilot model.


\subsection{Strategy}
We will run human subject experiments to test our hypotheses.
Over the past few years at UC Davis and NASA Ames, I’ve spent over 300 hours subject testing with more than 100 subjects.

\begin{table}[tb]
\centering
\caption{Previous Subject Testing Campaigns}
\label{table:mcruer1974a}
\small
\begin{tabular}{lll}
\toprule
Experiment   & Subjects & Length \\
\midrule
Lunar Lander~\cite{Karasinski2016, Karasinski2016Masters} & 30       & 3 hours \\
1D Tracking~\cite{Karasinski2016Masters} & 15       & 2.5 hours \\	
SAFER~\cite{Karasinski2016Masters, Karasinski2017}		   & 30       & 3 hours \\
3D Tracking  					   & 26       & 2 hours \\
\bottomrule
\end{tabular}
\end{table}

A draft journal article on the SAFER experiment is being prepared for submission in the journal of Human Factors.
A conference paper, ``Evaluating Augmented Reality in a Three-Axis Manual Tracking Task'', J. Karasinski and S. Robinson has recently been submitted for consideration at AIAA SciTech 2018.


In addition to these experiments, I have also completed numerous user tests for several products developed at NASA Ames.

The proposed research includes two human subject experiments and the development of a model to predict human performance.
Experiment One investigates if concurrent bandwidth feedback can be used to teach novice subjects to interpret depth cues in a three-axis manual tracking task
Experiment Two investigates if concurrent bandwidth feedback can decrease the required learning time to peak performance in a simulated robotic arm pick and place task.
The Model will extend Professor Hess’ structural model of the human pilot to include the effects of concurrent bandwidth feedback.
The outputs of this model will be compared to the results of Experiment One and Two.

Cool things we could do with The Model:
\begin{itemize}
\item Predict the optimal bandwidth provided task and controller characteristics
\item Predict the effects of scheduled bandwidth changes across training
\end{itemize}

Hypotheses:
\begin{itemize}
\item Concurrent bandwidth feedback can improve performance in simple, yet difficult tasks.
\item The depth cue offered by 3D augmented reality displays can improve performance.
\item Concurrent bandwidth feedback can be used to teach depth cues.
\item Concurrent bandwidth feedback can improve performance in robotic arm pick and place task.
\item The effects of concurrent bandwidth feedback on human performance can be modeled.
\end{itemize}

\end{document}
