%!TEX root = ../main.tex
\documentclass[float=false, crop=false]{standalone}
\usepackage[subpreambles=true]{standalone}
\begin{document}

\section{Proposed Research}
\subsection{Strategy}
We will run human subject experiments to test our hypotheses. Over the past few years at UC Davis and NASA Ames, I’ve spent over 300 hours subject testing with more than 100 subjects.

\begin{itemize}
\item Lunar Lander      30 subjects, 3 hours/subject
\item 1D Tracking       15 subjects, 5 days * 30 minutes/day
\item SAFER             30 subjects, 3 hours/subject
\item 3D Tracking       26 subjects, 2 hours/subject
\end{itemize}

In addition to these experiments, I have also completed numerous user tests for several products developed at NASA Ames.

The proposed research includes two human subject experiments and the development of a model to predict human performance. Experiment One investigates if concurrent bandwidth feedback can be used to teach novice subjects to interpret depth cues in a three-axis manual tracking task. Experiment Two investigates if concurrent bandwidth feedback can decrease the required learning time to peak performance in a simulated robotic arm pick and place task. The Model will extend Professor Hess’ structural model of the human pilot to include the effects of concurrent bandwidth feedback. The outputs of this model will be compared to the results of Experiment One and Two.

Cool things we could do with The Model:
\begin{itemize}
\item Predict the optimal bandwidth provided task and controller characteristics
\item Predict the effects of scheduled bandwidth changes across training
\end{itemize}

Hypotheses:
\begin{itemize}
\item Concurrent bandwidth feedback can improve performance in simple, yet difficult tasks.
\item The depth cue offered by 3D augmented reality displays can improve performance.
\item Concurrent bandwidth feedback can be used to teach depth cues.
\item Concurrent bandwidth feedback can improve performance in robotic arm pick and place task.
\item The effects of concurrent bandwidth feedback on human performance can be modeled.
\end{itemize}

\end{document}
