%!TEX root = ../main.tex
\documentclass[float=false, crop=false]{standalone}
\usepackage[subpreambles=true]{standalone}
\begin{document}
\section{Problem Statement}
\subsection{Motivation}
We aim to improve performance and decrease learning times for novice operators of highly complex motor control tasks in extremely dangerous environments.
We are specifically interested improving and modeling human performance in robotic arm tasks, which generally require a large amount of training to master.
The robotic arm on the International Space Station (ISS), for instance, requires hundreds of hours of time for astronauts to reach proficiency.
Being able to decrease this training time could lead to significant savings, and the predictive ability provided by modeling allows for much safer operation of the robotic arm.

Motor control tasks include a variety of skills such as playing tuba, pole jumping, and flying an aircraft.
These three examples have very little in common with each other, but it’s easy to recognize the difference between a novice and an expert in each case.
Humans rely on several kinds feedback during practice to improve their performance in a given motor control task.
These feedbacks can be largely grouped into two types: internal, or intrinsic feedback and external, or extrinsic feedback.
Internal feedback is anything a person can infer using their senses: the feel of the valve’s of the tuba as you play, your sense of balance mid-jump, or the sound the aircraft engine makes during an ascent.
Extrinsic feedback, on the other hand, is provided by an external source, often in the form of an expert instructor.

Extrinsic feedback comes in a variety of forms, and has a long history of improving performance in a large variety of motor control tasks.
This work will focus on a specific type of extrinsic feedback, which we call concurrent bandwidth feedback.
Concurrent feedback is provided in real-time, as an operator is completing a task.
Bandwidth feedback is provided when a particular value deviates outside a designated range or bandwidth.
Concurrent bandwidth feedback (CBF) is feedback provided to an operator in real-time during the time that a signal deviates out of a predefined range.
This type of feedback has been shown to improve performance in many simple motor control tasks, but has not been investigated in high degree of freedom, complex tasks.

% Training astronauts to better levels of performance can reduce the threat of disaster in an extremely dangerous environment.

\subsection{Research Aims}
We are interested in the measuring, modeling, and predicting the effects of concurrent bandwidth feedback (CBF) on human performance in complex manual control tasks.
To this end, this proposed research includes three research aims.
The first aim is complete, and the second aim is in progress.
\begin{enumerate}
\item Can CBF improve human performance in a 3DoF manual tracking task?
\begin{enumerate}
\item Can subjects use CBF to learn depth cues?
\item Do 3D augmented reality displays show improved performance compared to traditional 2D displays?
\end{enumerate}
\item Can CBF improve performance of simulated robotics tasks?
\begin{enumerate}
\item Can CBF reduce the required training time to peak performance?
\item Can CBF be removed after reaching peak performance with reducing subject performance?
\end{enumerate}
\item Can we develop a descriptive model of human performance which includes the effects of CBF? How does the model compare with experimental data?
\end{enumerate}

The remainder of this proposal is split into three sections: a literature review, the proposed research, and conclusions.

\end{document}
